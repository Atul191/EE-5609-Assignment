\documentclass[journal,12pt,twocolumn]{IEEEtran}

\usepackage{setspace}
\usepackage{gensymb}

\singlespacing

\usepackage[cmex10]{amsmath}
\usepackage{amsthm}
\usepackage{mathrsfs}
\providecommand{\sbrak}[1]{\ensuremath{{}\left[#1\right]}}
\providecommand{\lsbrak}[1]{\ensuremath{{}\left[#1\right.}}
\providecommand{\rsbrak}[1]{\ensuremath{{}\left.#1\right]}}
\providecommand{\brak}[1]{\ensuremath{\left(#1\right)}}
\providecommand{\lbrak}[1]{\ensuremath{\left(#1\right.}}
\providecommand{\rbrak}[1]{\ensuremath{\left.#1\right)}}
\providecommand{\cbrak}[1]{\ensuremath{\left\{#1\right\}}}
\providecommand{\lcbrak}[1]{\ensuremath{\left\{#1\right.}}
\providecommand{\rcbrak}[1]{\ensuremath{\left.#1\right\}}}
\usepackage{txfonts}
\usepackage{stfloats}
\usepackage{bm}
\usepackage{cite}
\usepackage{cases}
\usepackage{subfig}
\usepackage{longtable}
\usepackage{multirow}
\usepackage{mathtools}
\usepackage{steinmetz}
\usepackage{tikz}
\usepackage{circuitikz}
\usepackage{verbatim}
\usepackage{tfrupee}
\usepackage[breaklinks=true]{hyperref}
\usepackage{tkz-euclide} % loads  TikZ and tkz-base
%\usetkzobj{all}
\usetikzlibrary{calc,math}
\usepackage{listings}
    \usepackage{color}                                            %%
    \usepackage{array}                                            %%
    \usepackage{longtable}                                        %%
    \usepackage{calc}                                             %%
    \usepackage{multirow}                                         %%
    \usepackage{hhline}                                           %%
    \usepackage{ifthen}                                           %%
  %optionally (for landscape tables embedded in another document): %%
    \usepackage{lscape}     
\usepackage{multicol}
\usepackage{chngcntr}
\DeclareMathOperator*{\Res}{Res}
\renewcommand\thesection{\arabic{section}}
\renewcommand\thesubsection{\thesection.\arabic{subsection}}
\renewcommand\thesubsubsection{\thesubsection.\arabic{subsubsection}}

\renewcommand\thesectiondis{\arabic{section}}
\renewcommand\thesubsectiondis{\thesectiondis.\arabic{subsection}}
\renewcommand\thesubsubsectiondis{\thesubsectiondis.\arabic{subsubsection}}

\newcommand{\bignorm}[1]{\Bigl \| #1 \Bigr \| #1}
\newcommand{\norm}[1]{\| #1 \|}
% correct bad hyphenation here
\hyphenation{op-tical net-works semi-conduc-tor}
\def\inputGnumericTable{}                                 %%

\lstset{
frame=single, 
breaklines=true,
columns=fullflexible
}
\title{Assignment 1 EE5609}
\author{Lt Cdr Atul Mahajan\\ (AI20mtech13001) }
\date{September 2020}
\usepackage{natbib}
\usepackage{amsmath,amssymb,amsthm}
\newcommand{\myvec}[1]{\ensuremath{\begin{pmatrix}#1\end{pmatrix}}}
\let\vec\mathbf
\begin{document}
\maketitle
\begin{abstract}
This assignment finds the vector triple product.
\end{abstract}
Download python code from 
\begin{lstlisting}
https://github.com/Atul191/EE-5609-Assignment_1/blob/master/vectors.py
\end{lstlisting}
\section{Problem Statement}
\subsection{Find the value of:}
\begin{align*}
\myvec{1\\0\\0}^{T}
\brak{
\myvec{0\\1\\0}\times
\myvec{0\\0\\1}
}
+
\myvec{0\\1\\0}^{T}
\brak{
\myvec{1\\0\\0}\times
\myvec{0\\0\\1}
}
+\\
\myvec{0\\0\\1}^{T}
\brak{
\myvec{1\\0\\0}\times
\myvec{0\\1\\0}
}
\end{align*}
\subsection{Solution}
Equate these matrices as linearly independent vectors as a, b and c such as:
\begin{align}
\label{eq 1}
\vec{a}=\myvec{1\\0\\0}, 
\vec{b}=\myvec{0\\1\\0}, 
\vec{c}=\myvec{0\\0\\1}
\end{align}
Using scalar triple product property we deduce
\begin{align}
\vec{a}^{T}
\brak{
\vec{b}\times
\vec{c}
}=
\vec{b}^{T}
\brak{
\vec{c}\times
\vec{a}
}=
\vec{c}^{T}
\brak{
\vec{a}\times
\vec{b}
}
\end{align}
Note: Cross product is given by:
\begin{align}
\label{eq 3}
\myvec{a_1\\a_2\\a_3}\times
\myvec{b_1\\b_2\\b_3}=
\myvec{0&-a_3&a_2\\a_3&0&-a_1\\-a_2&a_1&0}
\myvec{b_1\\b_2\\b_3}
\end{align}
\subsection{Step 1}
Equating \eqref{eq 1} with problem statement we deduce the following:
\begin{align}
\vec{a}^T
\brak{
\vec{b}\times
\vec{c}
}+
\Vec{b}^T
\brak{
\vec{a}\times
\vec{c}
}+
\vec{c}^T
\brak{
\vec{a}\times
\vec{b}
}
\end{align}
\text{As Cross Product is anti-commutative we get:}
\begin{align}
\vec{a}^T
\brak{
\vec{b}\times
\vec{c}
}-
\Vec{b}^T
\brak{
\vec{c}\times
\vec{a}
}+
\vec{c}^T
\brak{
\vec{a}\times
\vec{b}
}
\end{align}
\begin{align}
=\vec{a}^{T}
\brak{
\vec{b}\times
\vec{c}
}-
\vec{c}^T
\brak{
\vec{a}\times
\vec{b}
}+
\vec{c}^{T}
\brak{
\vec{a}\times
\vec{b}
}
\end{align}
\begin{align}
\label{eq 7}
=\vec{a}^{T}
\brak{
\vec{b}\times
\vec{c}
}
\end{align}
\subsection{Step 2}
So instead of calculating each step we just calculate one iteration by referring \eqref{eq 3} and \eqref{eq 7} i.e.
\begin{align}
\myvec{0\\1\\0}\times
\myvec{0\\0\\1}=
\myvec{1\\0\\0}
\end{align}
\begin{align}
\implies
\myvec{1&0&0}
\myvec{1\\0\\0}=
\myvec{1}
\end{align}
\subsection{Step 3 (Answer)}
As the terms in the given problem are equal. so we have computed only one term as in step 2 and the solution to the problem statement looks like
\myvec{1}+\myvec{-1}+\myvec{1}=\myvec{1}
\end{document}
