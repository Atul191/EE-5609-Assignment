\documentclass[journal,12pt,twocolumn]{IEEEtran}

\usepackage{setspace}
\usepackage{gensymb}

\singlespacing

\usepackage[cmex10]{amsmath}
\usepackage{amsthm}
\usepackage{mathrsfs}
\providecommand{\sbrak}[1]{\ensuremath{{}\left[#1\right]}}
\providecommand{\lsbrak}[1]{\ensuremath{{}\left[#1\right.}}
\providecommand{\rsbrak}[1]{\ensuremath{{}\left.#1\right]}}
\providecommand{\brak}[1]{\ensuremath{\left(#1\right)}}
\providecommand{\lbrak}[1]{\ensuremath{\left(#1\right.}}
\providecommand{\rbrak}[1]{\ensuremath{\left.#1\right)}}
\providecommand{\cbrak}[1]{\ensuremath{\left\{#1\right\}}}
\providecommand{\lcbrak}[1]{\ensuremath{\left\{#1\right.}}
\providecommand{\rcbrak}[1]{\ensuremath{\left.#1\right\}}}
\usepackage{txfonts}
\usepackage{stfloats}
\usepackage{bm}
\usepackage{cite}
\usepackage{cases}
\usepackage{subfig}
\usepackage{longtable}
\usepackage{multirow}
\usepackage{mathtools}
\usepackage{steinmetz}
\usepackage{tikz}
\usepackage{circuitikz}
\usepackage{verbatim}
\usepackage{tfrupee}
\usepackage[breaklinks=true]{hyperref}
\usepackage{tkz-euclide} % loads  TikZ and tkz-base
%\usetkzobj{all}
\usetikzlibrary{calc,math}
\usepackage{listings}
    \usepackage{color}                                            %%
    \usepackage{array}                                            %%
    \usepackage{longtable}                                        %%
    \usepackage{calc}                                             %%
    \usepackage{multirow}                                         %%
    \usepackage{hhline}                                           %%
    \usepackage{ifthen}                                           %%
  %optionally (for landscape tables embedded in another document): %%
    \usepackage{lscape} 
\providecommand{\fourier}{\overset{\mathcal{F}}{ \rightleftharpoons}}
%\providecommand{\hilbert}{\overset{\mathcal{H}}{ \rightleftharpoons}}
\providecommand{\system}{\overset{\mathcal{H}}{ \longleftrightarrow}}
	%\newcommand{\solution}[2]{\textbf{Solution:}{#1}}
\usepackage{multicol}
\usepackage{chngcntr}
\DeclareMathOperator*{\Res}{Res}
\renewcommand\thesection{\arabic{section}}
\renewcommand\thesubsection{\thesection.\arabic{subsection}}
\renewcommand\thesubsubsection{\thesubsection.\arabic{subsubsection}}

\renewcommand\thesectiondis{\arabic{section}}
\renewcommand\thesubsectiondis{\thesectiondis.\arabic{subsection}}
\renewcommand\thesubsubsectiondis{\thesubsectiondis.\arabic{subsubsection}}

\newcommand{\bignorm}[1]{\Bigl \| #1 \Bigr \| #1}
\newcommand{\norm}[1]{\| #1 \|}
% correct bad hyphenation here
\hyphenation{op-tical net-works semi-conduc-tor}
\def\inputGnumericTable{}                                 %%

\lstset{
frame=single, 
breaklines=true,
columns=fullflexible
}
\title{Assignment 2 EE5609}
\author{Lt Cdr Atul Mahajan\\ (AI20mtech13001) }
\date{September 2020}
\usepackage{natbib}
\usepackage{amsmath,amssymb,amsthm}
\newcommand{\myvec}[1]{\ensuremath{\begin{pmatrix}#1\end{pmatrix}}}
\let\vec\mathbf
\begin{document}
\maketitle
\begin{abstract}
This assignment finds the values of vectors by Gaussian Row Elimination.
\end{abstract}
Download python code from 
\begin{lstlisting}
https://github.com/Atul191/EE-5609-Assignment/blob/master/Gaussian Elimination.py
\end{lstlisting}
\section{Problem Statement}
\subsection{Find the value of a,b,c and d:}
\begin{align}
\label{eq1}
\myvec{a-b&2a+c\\2a-b&3c+d}
=
\myvec{-1&5\\0&13}
\end{align}
\subsection{Solution}
Equate \eqref{eq1} in the form:
\begin{align}
\label{eq2}
\myvec{1&-1&0&0\\2&0&1&0\\2&-1&0&0\\0&0&3&1}
\myvec{a\\b\\c\\d}=
\myvec{-1\\5\\0\\13}
\end{align}
Using augmented matrix and row reduction on \eqref{eq2} we deduce in following steps
\begin{align}
\myvec{1&-1&0&0&-1\\2&0&1&0&5\\2&-1&0&0&0\\0&0&3&1&13}
\end{align}
Row elimination:
\begin{align}
\myvec{1&-1&0&0&-1\\2&0&1&0&5\\2&-1&0&0&0\\0&0&3&1&13}\\
    \xleftrightarrow[R_3\leftarrow R_3-2R1]{R_2 \leftarrow R_2-2R_1}
	\myvec{1&-1&0&0&-1\\0&2&1&0&7\\0&1&0&0&2\\0&0&3&1&13}\\
	\xleftrightarrow[R_4\leftarrow R_4-(-6)R_3]{R_3 \leftarrow R_3- \frac{-1}{2}R_2}
    \label{eq6}
	\myvec{1&-1&0&0&-1\\0&2&1&0&7\\0&0&\frac{-1}{2}&0&\frac{-3}{2}\\0&0&0&1&4}
\end{align}
Now performing back substitution on \eqref{eq6} we deduce the following:
\begin{align}
\vec{d}=4
\end{align}
\begin{align}
\frac{-1}{2}\vec{c}=\frac{-3}{2}
\end{align}
\begin{align}
\implies\vec{c}=3
\end{align}
\begin{align}
2\vec{b}+\vec{c}= 7
\end{align}
\begin{align}
\implies\vec{b}=2
\end{align}
\begin{align}
\vec{a}+\vec{b}= -1
\end{align}
\begin{align}
\implies\vec{a}=1
\end{align}
Solution:
\begin{align*}
\vec{a}=1,\vec{b}=2,\vec{c}=3,\vec{d}=4
\end{align*}
\end{document}
